% TEMPLATE for Usenix papers, specifically to meet requirements of
%  USENIX '05
% originally a template for producing IEEE-format articles using LaTeX.
%   written by Matthew Ward, CS Department, Worcester Polytechnic Institute.
% adapted by David Beazley for his excellent SWIG paper in Proceedings,
%   Tcl 96
% turned into a smartass generic template by De Clarke, with thanks to
%   both the above pioneers
% use at your own risk.  Complaints to /dev/null.
% make it two column with no page numbering, default is 10 point

% Munged by Fred Douglis <douglis@research.att.com> 10/97 to separate
% the .sty file from the LaTeX source template, so that people can
% more easily include the .sty file into an existing document.  Also
% changed to more closely follow the style guidelines as represented
% by the Word sample file. 
% This version uses the latex2e styles, not the very ancient 2.09 stuff.
\documentclass[letterpaper,twocolumn,10pt]{article}
\usepackage{usenix,epsfig,endnotes,xspace}
%\usepackage{babel}

\newcommand{\yad}{Lemon\xspace}
\newcommand{\oasys}{Juicer\xspace}

\newcommand{\eab}[1]{\textcolor{red}{\bf EAB: #1}}
\newcommand{\rcs}[1]{\textcolor{green}{\bf RCS: #1}}
\newcommand{\mjd}[1]{\textcolor{blue}{\bf MJD: #1}}

\begin{document}

%don't want date printed
\date{}


%make title bold and 14 pt font (Latex default is non-bold, 16 pt)
\title{\Large \bf Wonderful : A Terrific Application and Fascinating Paper}

%for single author (just remove % characters)
\author{
{\rm Russell Sears}\\
UC Berkeley
\and
{\rm Michael Demmer}\\
UC Berkeley
\and
{\rm Eric Brewer}\\
UC Berkeley
} % end author
% copy the following lines to add more authors

\maketitle

% Use the following at camera-ready time to suppress page numbers.
% Comment it out when you first submit the paper for review.
\thispagestyle{empty}


\subsection*{Abstract}

%\cite{nil} is a dummy citation to make bibtex happy.

\yad is a storage framework that incorporates ideas from traditional
write-ahead-logging storage algorithms and file system technologies,
while providing applications with increased control over its
underlying modules.  Generic transactional storage systems such as SQL
and BerkeleyDB serve many applications well, but impose constraints
that are undesirable to developers of system software and
high-performance applications, while filesystems provide limited
functionality to applications.

This paper generalizes write-ahead-logging algorithms, providing
applications with specialized functionality, cleaner semantics and
improved performance.

Applications may use our modular library of basic data strctures to
compose new concurrent transactional access methods, or write their
own from scratch.  This paper presents concrete low level examples
that modify the semantics of the buffer manager to reduce memory and
CPU overhead, reorder log entries for increased efficiency, and do
away with per-page LSNs in order to perform zero-copy transactional
I/O.  We argue that encapsulation allows applications to compose
extensions.

These ideas have been partially implemented, and initial performance
figures, and experience using the library compare favorably with
existing systems.


\section{Introduction}

\section{Existing transactional systems}

This section desribes DBMS systems, Berkeley DB and Database toolkits.
 
Relevant DB toolkit work (that I need to read): Exodus: E and ESM, Starburst,
Genesis, P2 (not ``Pier 2'').

\section{Write ahead logging}

This section desribes write ahead logging in generic terms, introduces
STEAL/no-FORCE and ARIES.

\section{Extensions}

This section desribes proof-of-concept extensions to \yad.
Performance figures accompany the extensions that we have implemented.

\section{Relationship to prior work}

This section describes how existing systems can be recast as
specializations of \yad.

\section{Conclusion}

\section{Acknowledgements}

\section{Availability}

Additional information, and \yad's source code is available at:

\begin{center}
{\tt http://\yad.sourceforge.net/}
\end{center}

{\footnotesize \bibliographystyle{acm}
\nocite{*}
\bibliography{LLADD}}

\theendnotes

\end{document}







